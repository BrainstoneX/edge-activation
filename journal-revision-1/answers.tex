\documentclass[11pt,a4paper]{article}
\usepackage{a4,amsfonts,amssymb,amsmath,latexsym,supertabular}
\usepackage[utf8]{inputenc}
\usepackage{graphicx}
\usepackage[usenames,dvipsnames]{color}
\usepackage[normalem]{ulem} 
\usepackage[arrow,matrix]{xy}
\usepackage{subfigure}
\usepackage[section]{placeins}

\usepackage{multirow}

\usepackage[USenglish]{babel}
% \selectlanguage{\USenglish}

\def\textfraction{.01}
\def\topfraction{.99}
\def\bottomfraction{.99}
\def\floatpagefraction{.99}

\setlength{\textheight}{23.8cm}
\addtolength{\oddsidemargin}{-0.5cm}
\setlength{\textwidth}{16.0cm}
\setlength{\parskip}{1.0ex plus 0.5ex minus 0.7ex}
\setlength{\parindent}{0em}
\setlength{\voffset}{-1.1cm}



\newcommand{\Cal}[1]{{\mathcal {#1}}}

\usepackage{amsmath}
\usepackage{amsthm}

\usepackage{mathtools}

\usepackage{enumerate}

\usepackage{hyperref}

\newcommand{\cU}{\mathcal{U}}
\newcommand{\X}{\mathcal{X}}
\newcommand{\R}{\mathbb{R}}

\newcommand{\marc}[1]{\textcolor{blue}{#1}}
\newcommand{\stefan}[1]{\textcolor{orange}{#1}}
\newcommand{\lasse}[1]{\textcolor{Plum}{#1}}
\newcommand{\todo}[1]{\textcolor{red}{#1}}

\begin{document}

\section*{Answers to the Reviewers}

\textit{We thank the three reviewers for their constructive and helpful feedback. We have aimed to make changes to our paper that account for all of the points raised. We respond to each of the reviewers' comments individually, and our responses are written in italics below. All the changes made to the paper are highlighted in \textcolor{blue}{blue}.}

\section*{Editor}


I have received the reports from our advisers on your manuscript, "Non-preemptive tree packing", which you submitted to Algorithmica.

Based on the advice received, I have decided that your manuscript could be reconsidered for publication should you be prepared to incorporate major revisions.
When preparing your revised manuscript, you are asked to carefully consider the reviewer comments which are attached, and submit a list of responses to the comments. Your list of responses should be uploaded as a file in addition to your revised manuscript.


\textit{We thank the editor for the opportunity to revise the paper. As can be seen below, all points were carefully addressed.}


\section*{Reviewer 1}


This paper studies the non-preemptive tree packing problem in which the input is an undirected edge-weighted graph $G$ and the goal is to activate every edge $e$ of $G$ for some time interval of length $w(e)$ such that the activated edges keep $G$ connected for the longest possible time. The weights of edges and time intervals are positive integers. It is shown that this problem is strongly NP-hard even on graphs of treewidth 2, and it does not admit a PTAS. It is also shown that the problem is NP-hard even if the optimal objective value (the time interval that G is connected) is at least 7. Moreover, a polynomial-time algorithm is presented for the case where optimal objective value is at most 3.

\textbf{Major comments}

This paper is a collection of nice results on the same problem and deserves publication. There are two major issues that I want the authors to address:

\begin{itemize}

\item[1.] The presentation of the paper needs to be improved. Many technical terms are used without being defined, for example, bandwidth, outerplanar graphs, series-parallel graphs, strongly NP-hard; for someone who is not familiar with this concepts it might be hard to follow the arguments in the paper. Below I provided a collection of minor comments; this is just a sample and do not cover all required changes. I recommend the authors to edit and proof-read the paper carefully.

\textit{We agree that we were a bit careless in leaving out the definitons of these terms. We now define all graph-theoretic notions in the new section 2.3. The notion of strongly NP-hardness is given in the introduction.}

\item[2.] Lemma 1 is used (in many places) without being proved. The proof is left as an easy task to the reader but it is hard for me to verify this lemma. A solid proof of this lemma is important because it is at the heard of Theorems 4, 8, 9, 10.
The statement of the lemma also need to be reworded. Although it is one sentence, it makes two claims
\begin{itemize}

\item[(i)] If ntp = $T$ (the $T$ intervals are not necessarily connected) then you can shift/exchange activation times to get the same ntp but with continuous/consecutive intervals.

\item[(ii)] We can make the connected graph (during the above interval) to be a spanning tree.

\end{itemize}
Both above claims need proof.


\textit{Thank you for pointing out this issue. It turned out that a completely formal proof of Lemma 1 is more lengthy then we imagined, so we think it is a good addition to the paper and helps the reader to get comfortable with the concepts and notation used later on. We reworded the statement of the lemma in the suggested way.}

\item[3.] The following is a semi-major issue: Again Lemma 2 is used without being proved. Please provide a solid proof for this lemma.

\textit{We provided a full proof of Lemma 2.}

\end{itemize}




\textbf{Minor comments}

\begin{itemize}

\item A linear program is introduced at the beginning of the paper while its parameter $x_T$ has not been introduced until later. It would be better to introduce $x_T$ before or immediately after the linear program. 
Also, the linear program that is introduced here, has not been used anywhere else in the paper. I would recommend to remove this linear program (which mainly distracts and confuses the reader) but instead state the problem in a few sentences with simple words.

\textit{The initial motivation behind the decision to present the problem in this LP-form is the paper of Barahona (Packing spanning trees, reference [2]), which starts exactly the same way. However, you are correct to point out that in contrast to Barahona's paper, our paper does not require further reference to the variable $x_T$. We followed your suggestion and replaced the LP with a simple-word description and an additional image. }

\item At some point in the intro (where you introduced the problem) it has to be clarified whether the goal is to have a connected graph $G$ or a spanning tree of $G$. For example if we activate all edges of the graph in Fig 1 at the same time, do we get a valid answer within an interval of length 2? 

\textit{ Indeed, we agree that it has to be specified whether a connected graph or a spanning tree is required. This is already the case, because in our definition, we state that only connectedness is required.  However, we agree that the reader might be confused and think that a spanning tree is required. It is important to note that due to Lemma 1 both of these choices of definition are equivalent.  We want to keep the introduction short and believe that the right point to adress details such as these is Section 2.1 Formal Problem Definition. We added a clarifying paragraph to Section 2.1. We also added an additional paragraph to Section 2.2 to explain in detail how the \emph{preemptive} scheduling is equivalent to Nash-Williams problem.}

\item page 2, lines 40-41: What if we activate two edges at time 1?

\textit{You are right, we forgot to mention this case. Because another reviewer was also confused by this sentence, we decided to change it and mention at this point in the text only the possibility that two edges are activated at time 0. The formal defintion will be made clear in Section 2.1. }

\item page 3, line 43: Is $t$ integer?

\textit{Yes, added clarification.}

\item page 3, line 49: induced subgraph of $G$ by $V'$

\textit{Changed as proposed.}

\item page 3, line 49: by removing a vertex $v$ from $G$

\textit{Changed as proposed.}

\item page 3, line 53: Here integer weights are considered. It would be better to say something about real weights, and whether or not it has been studied earlier. Are there some previous works on real weights?

\textit{We only consider integer weights in our paper. Note that if the weights are rational, they can be multiplied by the greatest common demoninator to yield an equivalent integral instance, which is larger by at most a polynomial factor. If the weights are real-valued and encoded in finite input length (for example algebraic numbers), we are not sure if the problem becomes more interesting -- Lemma 1 still works in a similar way, which implies that the problem stays in the class NP. So the problem remains to be NP-complete, but the proofs and details become more technical. We added these observations to Section 2.1. We are not aware of any prior works with real-valued weights.}

\item page 4, line 7: Shouldn't $t$ be at least $\sigma(e)$, because $e$ activates at time $\sigma(e)$? Why it is $t \geq \sigma(e)+1$?

\textit{No, this is correct. If the graph in total is connected for, say, $\beta$ time units, then it is connected for the whole time interval $[0, \beta]$. This means the the $t$-th time slot is given by the interval $[t-1,t]$ for each $t=1,\dots,\beta$. The edges active in the $t$-th time slot are therefore those edges with both $\sigma(e) \leq t-1$ and $\sigma(e) + w(e) \geq t$, or equivalently, $\sigma(e) + 1 \leq t \leq \sigma(e) + w(e)$.}

\item page 4, line 11: You refer to "the number of time slots" here; maybe clarify that the time slots (in which $G$ is connected) need not be continuous/consecutive.

\textit{We agree that such a hint would be helpful to the reader. We added a clarification to Section 2.1.}

\item page 4, line 31: Better to rewrite as "such that the numbers in every triplet sum up to $Q$".

\textit{Changed as proposed.}

\item page 4, line 33-35: Here the authors mentioned series-parallel (rep. outerplanar) graphs, while the following theorems use $K_{2,\ell}$ and (rep. graphs of bandwith 2). Maybe clarify that $K_{2,\ell}$ is series-parallel and bandwidth 2 is outerplanar.

\textit{We rewrote the paragraph, after reviewer 3 pointed out that every outerplanar graph is subgraph of a series-parallel graph. We now distinguish between $K_{2,\ell}$ (which has unbounded bandwidth) and graphs of bandwidth 2.}

\item page 4, line 53: "every edge is contained in $[0, \beta]$" Here Lemma 1 is used without being referred to.

\textit{This is a subtle issue. This fact does not follow from Lemma 1. Instead, it follows from the fact that the sum of all edge weights is exactly $\beta$ times the weight of a spanning tree (if not every activity interval would be contained in the interval $[0, \beta]$, there would not be enough weight anymore to keep the graph connected for $\beta$ time slots). Here is an example why such a fact does not follow from Lemma 1: Consider again the example from the introduction, the complete graph on three vertices. The edge weights this time are $w(e_1) = w(e_2) = 2$ and $w(e_3) = 10$. In this example we have $\beta = 2$. Even though the statement of Lemma 1 is true, there does not exist a schedule such that all activity intervals of all the edges are contained in $[0, \beta]$. We reworded the sentence you pointed out to make this logic mroe clear.}

\item page 4, line 55: "It is easy to see that", why? please clarify.

\textit{We added a clarification.}

\item page 7, lines 27-28: I think it is necessary to mention the details/proof here even if this is done by routine arguments? What do you mean by routine arguments? This is a journal paper (with no page limit) so please clarify and add necessary details.

\textit{We added a proof of Lemma 2.}

\item page 7, line 33: "so-called special edge $\{u, z\}$" What is this edge? What is special about this edge? Where it was introduced? any reference? 

\textit{This is merely a name. We call this edge the \emph{special edge}, so that it is easier to talk about it in the proof. The algorithm returns a tuple of a graph $H$ and an edge $\{u,z\}$ in the graph. This edge has the property, that either there is a Hamilton cycle using it, or there is no Hamilton path starting at $u$. This edge is then called the special edge. We reworded the statement of Lemma 2 in order to improve its clarity.}

\item page 7, line 55: "also" does not fit here

\textit{The word "also" was used to indicate that the graph $G$ restricted to the vertex set $U \cup Z$ has the same edges as the graph $H$, but we agree that the formulation was a bit unclear. We reworded the statement to be more clear.}

\item page 8, fig 3: Labels are large; better to have them of the same size as the main text.

\textit{Changed as proposed.}

\item page 10, line 10: ??

\textit{Should be Table 1. Fixed.}

\item page 12: Please clarify the connection of matroids to connectivity of the final graph G in time interval $[0,3]$.

\textit{We added a thorough explanation of the relation between matroid bases and the connectivity of $G_t$ for $t=1,2,3$.}

\item page 12, line 43: "It is easily seen" Please clarify the argument with valid reasoning.

\textit{We clarified the argument further.}

\item Page 13, line 39: folklore result [8]: There is a very simple counting argument for this claim. Please mention this counting argument in 1-2 sentences. Perhaps it is not necessary to mention a reference for this.

\textit{In order to see $\text{ntp}(K_n, w) = \lfloor n/2 \rfloor$, we need to prove that the number of spanning trees that can be packed into $K_n$ is $\geq \lfloor n/2 \rfloor$ and $\leq \lfloor n/2 \rfloor$. But to the best of our knowledge the counting argument is only a proof for the upper bound. We therefore decided to keep this sentence as is.}

\item page 13: Theorem 6: The statement of this theorem is a bit unusual. Please rephrase, or maybe just say that greedy solves the problem only on cactus graphs.

\textit{We would like to keep this statement, as it is not merely a statement that Greedy solves the problem on cactus graph, but additionally that the class of cactus graphs is characterized by the property that Greedy is optimal. We think this is a nice (if not very hard to obtain) result. We are not sure in which way we should rephrase the statement, as it already exactly states what we want to say, i.e.\ that item (i) and (ii) are equivalent.}

\end{itemize}


\section*{Reviewer 2}

The authors introduce a Non-preemptive Tree packing problem, where given an edge-weighted graph $G$, the goal is to find a schedule for the edges or to assign each edge to a start time (the edge belongs to $G$ only during the timeslot [start time+1, start time+its weight]) such that $G$ remains connected for the longest possible overall time. They obtain several hardness and algorithmic results:

\begin{itemize}
\item NP-hard in $K_{2,n}$: a simple reduction from 3-PARTITION
\item NP-hard on graphs of bandwidth 2: again a simple reduction from 3-PARTITION
\item NP-hard to decide if there is a schedule of value $\geq 7$: a more interesting reduction from Hamilton cycle in 3-regular graphs

\item Polytime for objective value $\geq 3$: using a very simple scheme of 3-matroid partitioning    
\item $(n-1)$-approximation using a greedy strategy which activates the edges of the largest possible weight
\item Optimal in cactus graphs using the same greedy strategy
\item Exact FPT algorithm with parameter treewidth + maximum edge weight/feedback edge set size/ $k$, where $m-k$ edges have weight 1 and $k$ edges have weight at most $k$
\end{itemize}

\textbf{Major comments}
\begin{itemize}

\item What is the real-life application or motivation behind this problem? 

\textit{Thank you for pointing out this issue. As was also pointed out by another reviewer, our paper lacked motivation. This is due to the fact that the nature of our work was exploratory. However, relations to other areas of mathematics exist. We added a paragraph to the introduction, explaining how our problem relates to the different areas of scheduling, time-varying graphs, and tree packing with consecutivity constraints.}

\item I have found the hardness results interesting enough, it might be possible to find more non-trivial algorithmic results in generalized settings 

\textit{This is indeed likely to be true. However, we do not want to further extend the scope of the paper, as we believe that it already contains a large number of results.}

\item can one obtain an $O(1)$ approximation in the general case? On the other hand, even in the simple case of 1-2 edge weights, the problem is not well-researched

\textit{Indeed, the open question if there is an $O(1)$-approximation in the general case is probably a harder question than to understand if the problem in the 1-2-edge weight case admits a poly-time algorithm. Still, it is not the case that one statement implies the other, so the questions are independent of each other. As we are the first people to consider non-preemtive tree packing, both problems are open. We believe that both open problems are interesting to state.}


 
\item Page 2: " or activate the three edges...0,1,2" Don't you have to activate two edges at time 0 to make the graph connected during $[0,1]$?

\textit{We wanted to express that the set of times where the graph is connected does not necessarily have to be an interval of consecutive time slots which starts at 0 (however, this distinction is not too important, as it is equivalent to only consider consecutive intervals or not). Because this also confused another reviewer, we have decided to not mention this technical detail at this point of the introduction. Instead, we have added a paragraph to Section 2.1 Formal Problem Definition explaining the relationship in more detail. We believe that this makes the paper more clear overall.}

\item Lemma 1: use a different symbol than $T$, which was used before to denote a tree

\textit{Changed to $\beta$.}

\item It would be better to justify the choice of the parameters for the parameterized algorithms, especially for the last two results which look fairly artificial

\textit{As the problem is already NP-hard even for very highly restricted graphs, considering all the usual parameters from parameterized complexity do not result in efficient algorithms. Therefore it is expected that the considered parameters are a bit exotic. As proposed, we added this justification to the text.}

\item Are there any connection between machine scheduling and the problem studied, especially is it possible to use results from the wealth of literature on scheduling?

\textit{Even though we searched thoroughly, and we have some knowledge about the area of scheduling problems, we could not find any applicable results from this area. It is of course possible that there exist some, but that we could not find them.}

\item LP rounding has become very handy for approximating scheduling problems. Is it possible to use LP rounding to obtain improved results?

\textit{We are not experts in the area of LP-rounding, so we unfortunately do not have the necessary knowledge to answer the question. But it is definitely an interesting question for further research. It should be stated that the IP of the problem is quite complicated, due to the interaction of requiring connectedness and non-preemption.}

\item What is the best approximation possible in planar, bipartite graphs? 

\textit{We do not know, this is an open research question. Note that every graph can be made bipartite by subdividing an edge of cost $w$ into two edges of cost $w$ and $C$, where $C$ is a large constant. This instance is equivalent to the previous instance. So the relevant question is, what is the best approximation possible in planar graphs? }

\end{itemize}





\section*{Reviewer 3}

The paper presents a novel problem, in which a graph with integral edge-weights
is given, and the task is to create longest possible schedule of time slots. In
each time slot some edges must be selected such that the subgraph induced by
the selected edges is connected. Additionally, each edge must be selected
during an interval of consecutive time slots, and the interval cannot be longer
than is the edge's capacity.

The problem is presented as a non-preemptive version of the tree-packing
problem, in which the goal is to maximize the number of selected spanning
trees, provided that the weight of an edge is the upper bound on the number of
trees in which the edge may occur (i.e. the edge does not need to be selected
in consecutive time steps). 

Both these problems have a number of motivations from different areas: they may
be seen as a packing problems in graphic spaces, or scheduling problems where
the connectivity is a structurally rich global constraint, or in the setting of
so called time-varying graphs and so on. It is unfortunate that the authors
don't provide a broader context for the research, and limit themselves to the
technical contribution. This feeling that the paper is a slightly expanded
technical conference paper, is prevailing also in other parts of the paper.

\textit{Thank you for pointing out this issue and for pointing out which areas could be related to our problem. We also agree that the missing motivation and connection to other areas are the weakest points of the paper. This is due to the fact that the nature of our work was exploratory. However, relations to other areas of mathematics exist. We added a paragraph to the introduction, explaining how our problem relates to the different areas of scheduling, time-varying graphs, and tree packing with consecutivity constraints.}

The technical contribution is in investigating the hardness of the problem.
Compared to the original tree-packing problem, which is polynomial-time
solvable, adding the non-preemptiveness constraint makes the problem much
harder.  In particular, the authors show that it is NP-hard even on complete
bipartite graphs $K_{2,n}$, and on graphs with bandwidth 2. Also, it is NP-hard to
decide whether the optimal value is at least 7, which immediately yields the
lower bound 7/6 in the approximation ratio (unless P=NP). These results are
complemented with some positive results: whether the optimal value is at least 3
can be decided in polynomial time. The natural greedy algorithm has
approximation ratio linear in the size of the graph, and the problem is FPT if
parameterized by treewidth and maximal edge weight.

The proofs use standard techniques, the NP-hardness is proven by reduction from
(a variant of) partition, and Hamiltonicity. The edge weights play crucial role
in the analysis: on one hand, heavy edges can be scheduled from the beginning
without loss of generality, on the other hand, the overall sum of weights
provides upper bounds on the length of the schedule.  The FPT result is a
consequence of Courcelle's theorem.

Overall, the paper provides a detailed picture of the complexity of an
interesting problem which features global connectivity requirements over time.
Studying this type of constraints may be of interest in several research
areas.

\textbf{Major comments}

The paper is well written, and the presented technical parts are well
explained.  Apart from the lack of broader context in the introduction, there
are also some places with omitted proofs. This practice would be understandable
in a conference paper, but in a journal contribution it is debatable. In
particular:

\begin{itemize}

\item I agree that Lemma 1 is trivial, but since it is the first technical
  statement of the paper, which is extensively used later, I believe it would
  help the reader to get familiar with the model if a proof was presented. If
  you decide not to prove it, then at least rename it to "Observation", "Fact",
  or similar.

\textit{We provided a proof of Lemma 1.}

\item I would definitely add the proof of Lemma 2. Again, it can be agreed that it
  is not the most involved or thrilling part of the paper, and the reader
  should be able to fill in the details, but I believe that the level of
  details omitted in the proof is too high for a journal paper.

\textit{We provided a proof of Lemma 2.}

\item Similar situation is in Theorem 7, albeit there I am not sure whether the
  full formulation of the "spanning tree" predicate in MSO would actually
  increase the readability. However, at least some reference to a general
  source of those "routine methods" would be nice; the aim is to make it
  possible also for readers from different areas to check the correctness of
  the proofs.
  
\textit{We agree that a full formulation in MSO would rather increase confusion than reduce it. We followed your advice and inserted a reference to chapter 7.4 of the excellent book  "Parameterized Algorithms" by Cygan et al. } 
 
\end{itemize}

\textbf{Minor comments}

\begin{itemize}

\item The text before Theorem 1 should be reformulated. As such, it seems that
Theorem 1 is made obsolete by Theorem 2 (since outerplanar graphs are
series-parallel).  However, Theorem 1 gives the NP-hardness specifically for
K(2,n), and Theorem 2 for graphs with bandwidth 2 (again, graphs with bandwidth
2 are outerplanar, but outerplanar graphs in general have unbounded bandwidth).
Also K(2,n) has unbounded bandwidth so the theorems are indeed independent.

\textit{Thank you for pointing out this issue -- we overlooked the fact that every outerplanar graph is a subgraph of a series-parallel graph. We rewrote the paragraph to distinguish between $K_{2,\ell}$ and bandwidth-2 graphs.}

\item p2.  l46: the problems are computationally hard, but "devastating" is maybe an overstatement
            
\textit{When we presented this paper at a conference, some other attendee specifically told us they enjoyed this choice of words. We argue that this is a matter of taste and would like to keep the wording.}            
            
\item p10. l10: ?? should be Table 1

\textit{Fixed.} 

\item p10. l47: Don't start the first sentence of the proof with "So".

\textit{Fixed.}

\item Theorem 7: If both treewidth and edge weights are bounded, then the algorithm
  is actually polynomial. It's FPT algorithm with respect to those parameters.
  
\textit{Thank you for pointing out this issue. We reworded the statement of the theorem accordingly.}
  
\item p15. l56: I see what you mean by "close to preemptive", but the reason here is 
  that (most of) the edges can be used only one time unit.
  
\textit{If the graph is simple, your assessment is correct. However, the proof of the theorem easily carries over to multigraphs. In a multi-graph, if there are $t$ edges of weight 1 between two vertices, this models a situation of an edge of weight $t$ which allows for preemption. Therefore this lemma indeed makes a statement about instances that are close to the preemptive case. We would like to avoid having to talk about multigraphs in the paper, because it introduces additional technical difficulties. Also notice that a multi-edge can be easily modelled by subdividing an edge of weight $w$ into two edges of weight $w$ and $C$, where $C$ is a large constant (However, this is not applicable in the context of Theorem 10, because the edge weights are bounded). }
  
\end{itemize}

\end{document}
